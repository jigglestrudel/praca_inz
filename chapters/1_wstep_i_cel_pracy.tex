\chapter{Wstęp i cel pracy (Tomasz Krępa)}
\label{chap:introduction}
W ostatnich latach przemysł samolotów bezzałogowych stał się dostępny dla o wiele szerszej grupy użytkowników. Drony znajdują swoje zastosowanie już nie tylko w przemyśle zbrojeniowym, ale także przemyśle rolniczym, leśnictwie i celach rozrywkowych. Jednym z możliwych sposobów wykorzystania statków bezzałogowych w agrokulturze jest automatyzacja pozyskiwania informacji na temat stanu gleby. W tym celu stosowane są samoloty autonomiczne typu fixed-wing, które mają większe możliwości w zakresie odległości i czasu lotu niż standardowe drony, jednak nie są w stanie wykonywać takich samych manewrów co standardowe drony wykorzystywane komercyjnie, np. w fotografii. Obecnie ciężko jest znaleźć rozwiązania pozwalające na przystępne wyznaczanie obszaru i trajektorii przelotu samolotów tego typu. Większość implementacji nie skupia się w wystarczającym stopniu na wyznaczaniu obszaru i pozostawia tę część rozwiązania użytkownikowi. 

Motywacją do realizacji tego projektu jest możliwość rozwoju technologii wspierającej agrokulturę i leśnictwo. Automatyzacja w tak ważnych ekologicznie obszarach jest w dzisiejszych czasach bardzo istotna. Możliwość zoptymalizowanego badania stanu gleby rolnej lub wilgotności ściółki leśnej pozwoli na usprawnienie procesów hodowli roślin czy informacji przeciwpożarowej. Patrząc na pozytywne skutki wykorzystania samolotów bezzałogowych w tych obszarach, można niezaprzeczalnie stwierdzić, że jest to technologia zasługująca na rozwój.

Celem niniejszego projektu dyplomowego jest stworzenie aplikacji internetowej pozwalającej na planowanie trasy przelotu drona na podstawie satelitarnych zdjęć terenu. Zaimplementowane zostaną algorytmy wykrywania krawędzi oraz wytyczania krzywych zakreślających zadany obszar. Aplikacja zostanie wykonana przy wykorzystaniu środowisk Next.js i Python.

W drugim oraz trzecim rozdziale omówione zostały zagadnienia zgłębiające dalej dziedzinę, której poświęcamy naszą pracę. Autorem rozdziału czwartego jest XX, w którym przedstawia on aktualny stan wiedzy i technologii dotyczącej naszego rozwiązania. Autorami rozdziału piątego są YY i XX. Sekcja ta poświęcona jest analizie naszego systemu. Podrozdziały napisane przez YY są poświęcone dokładnej specyfikacji postawionych przed nami wymagań projektowych, a pozostałe podsekcje, autorstwa XX, przedstawiają nasz zespół, harmonogram prac nad implementacją oraz analizują ryzyka związane z realizacją przedsięwzięcia informatycznego. W szóstym rozdziale, XX opisał projekt naszego systemu, a w szczególności przeznaczenie użytkowania produktu, koncepcje naszych rozwiązań a także projekty poszczególnych komponentów wykorzystanych w finalnym narzędziu. Autorem siódmego rozdziału jest XX i ta część pracy poświęcona jest implementacji a także testom i weryfikacji możliwości naszej aplikacji w środowisku docelowym. Rozdział ósmy, którego autorem jest XX, poświęcony został naszym wynikom i wnioskom wyciągniętych w trakcie realizacji projektu. W tej części zawarte zostały także możliwości i pomysły dalszego rozwoju narzędzia. 